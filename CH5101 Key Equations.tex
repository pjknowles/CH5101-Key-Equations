\documentclass[a4paper, 11pt, fleqn]{article}
% for learning
\def\title{CH5101 Foundations of Physical Chemistry --- Key Equations}
% for exam paper
\def\title{Appendix 2: Key Equations}\pagestyle{empty}

\usepackage[a4paper,  margin=1.2cm, bottom=2.5cm]{geometry}
\usepackage[utf8]{inputenc}
\usepackage{cmbright}

\usepackage{longtable}
\usepackage{color}
\DeclareRobustCommand{\ltsym}{{\mbox{%
\def\width{0.8pt}
\hbox to \width{\relax}%
\hbox to 0pt{\hbox to 0pt{\hss\(\circ\)\hss}%
\hbox to 0pt{\hss\(-\)\hss}%
}
\hbox to \width{\relax}%
}}}

\def\titlebanner{\relax}
\usepackage{ifthen}
\usepackage{amsmath,amstext}

%

% \renewenvironment{note}[0]{\relax}{\relax}
\def\Text#1{\text{\color{textcolor}#1}}
\definecolor{mathcolor}{rgb}{0,0,0}
\definecolor{textcolor}{rgb}{0,0,0}
\definecolor{memorise}{rgb}{0,0,0}
\definecolor{explain}{rgb}{0.0,0,0.0}
\usepackage{array}
\newenvironment{explain}{\color{explain}\def\arraystretch{1.6}\setlength{\tabcolsep}{.05\textwidth}\begin{longtable}{@{}p{.35\textwidth}%
p{.55\textwidth}%
}\raggedright}{\end{longtable}}
\newcolumntype{E}{>{$\displaystyle}p{.55\textwidth}<{$}}
\newenvironment{concepts}{\color{explain}\def\arraystretch{2.5}\setlength{\tabcolsep}{.05\textwidth}\begin{longtable}{@{}p{.35\textwidth}
% p{.55\textwidth}%
E
}\raggedright}{\end{longtable}}
\begin{document}
% \maketitle
\begin{center}
\textbf{\Large \title}
\end{center}

%\clearpage
%\addtocounter{section}{-1}
\section*{Kinetics}
\begin{concepts}Integrated rate law for first order reaction & [A]=[A]_0 \, e^{-k t} \\
    Integrated rate law for second order reaction & [A]=\frac{[A]_0 }{ 1+k [A]_0 t}\\
    Half life & t_{1/2} = \frac{\ln 2 }{ k}
    \\
    Arrhenius equation & k=A\, e^{-E_a/RT}
\end{concepts}
\section*{Spectroscopy}
\begin{concepts}
    & c=\nu\, \lambda
    \\
    Planck-Einstein & E=h\,\nu = \frac{h\,c}{\lambda} = h\, c\, \bar{\nu}
    \\
    Beer-Lambert law & A=-\log_{10}\frac{I}{I_0} = -\log_{10} T = \epsilon\,c\,l
    \\
    Rydberg equation & \bar{\nu}=R_{\text H}\left(\frac1{n_1^2}-\frac1{n_2^2}\right)
    \\
    Vibrational energy levels & E_v=h\,\omega(v+\tfrac12),\quad \omega=\frac{1}{2\pi}\sqrt{\frac k\mu}
    \\
    Rotational energy levels & E_J=h\,B\,J(J+1), \quad B=\frac h{8\pi^2\, c\, I}, \quad I=\mu\,r^2
\end{concepts}
\section*{Gases}
\begin{concepts}Ideal gas equation of state & p\,V=n\,R\,T
\\
Van der Waals equation of state & \left(p+\frac a{V_m^2}\right)\left(V_m-b\right)=R\,T
\\
Partial pressure in an ideal gas & p_{\text X} = p_{\text{tot}}\, \frac{n_{\text X}}{n_{\text{tot}}}
\\
Compression factor &Z=\frac{p\,V}{n\,R\,T}
\\
Most probable particle speed & v_{\text{mp}} = \sqrt{\frac{2k_B T}{m}}
\\
Mean particle speed & v_{\text{mean}} = \sqrt{\frac{8k_B T}{\pi m}}
\\
Most probable particle kinetic energy & E_{\text{mp}} = \tfrac12 k_B T
\\
Mean particle kinetic energy & E_{\text{mean}} = \tfrac32 k_B T
\\
Reduced mass & \mu=\frac{m_{\text X} m_{\text Y} }{ m_{\text X} +m_{\text Y} }
\\
Collision frequency & z=\sigma\,\mathcal{N}\sqrt{\frac8{\pi\mu k_B T}}
\\
Collision density & Z_{\text X \text X}=2\sigma\,\mathcal{N}^2\sqrt{\frac{k_B T}{\pi\mu }}=\frac{2\sigma\,p^2}{\sqrt{\pi m k_B^3 T^3}}
\\
& Z_{\text X \text Y}=2\sqrt 2\,\sigma\,\mathcal{N}_{\text X} \mathcal{N}_{\text Y} \sqrt{\frac{k_B T}{\pi\mu }}=\frac{2\sqrt 2\,\sigma\,p_{\text X}p_{\text Y}}{\sqrt{\pi m k_B^3 T^3}}
\\
Mean free path & \lambda = \frac1{\sqrt2\,\mathcal N\,\sigma}=\frac{k_B T}{\sqrt2\,p\,\sigma}
\\
Diffusion coefficient & D=\frac2{3\mathcal N \sigma}\sqrt{\frac{k_B T}{\pi\mu}}
\\
Collisions with walls & Z_{\text W}=\frac p{\sqrt{2\pi m k_B T}}
\end{concepts}
\section*{Thermodynamics}
\begin{concepts}First law
&
\Delta U = \Delta q + \Delta w
\\
Mechanical work
&\color{memorise} \delta w = \text{force} \times \text{distance}
= -p\,\delta V
\\
&\color{memorise} \Delta w = -\int p\,dV, =- p\,\Delta V\text{ if $p$
  constant}
\\
Sample heat capacity & 
\color{memorise}
C = \frac{dq}{dT}
\\
Constant-volume sample heat capacity & 
\color{memorise}
C_V = \left.\frac{\partial U}{\partial T}\right)_V
\\
Enthalpy
&
\color{memorise}
H = U + p\, V
\\
Enthalpy change at constant $p$& \Delta H = \Delta q=\Delta U + p\,\Delta V
\\
$pV$ contribution to enthalpy change of reaction with ideal gas
components
&
\Delta H = \Delta U + \Delta (p\,V) = \Delta n\, R\,T
\\
Heat capacity at constant pressure
&
\color{memorise}
C_p = \left.\frac{\partial H}{\partial T}\right)_p
\\
% Condensed-phase heat capacity at constant pressure
% &
% $C_p\approx C_V$ because volume does not change much with temperature
% \\
Hess's Law
%, arising from the fact that $H$ is a state function
&
\color{memorise}
\Delta H^\ltsym = \Delta_f H^\ltsym[\text{products}]
- \Delta_f H^\ltsym[\text{reactants}]
\\
Entropy
&
S = k\,\ln W ;
\color{memorise}
\quad \delta S = \frac{\delta q_{\text{reversible}} }{ T}
\\
Second law
% & Entropy of an isolated system can never decrease
% \\
% System/surroundings version of second law
&
\color{memorise}
\Delta S_{\text{universe}} = \Delta S_{\text{system}} + \Delta S_{\text{surroundings}}
\ge 0
\\
Entropy change of expanding ideal gas
&
\color{memorise}
\Delta S = \frac{\Delta q_{\text{rev}}}{T} = - \frac{\Delta w_{\text{rev}}}{T} = n \, R
\, \ln \frac{V_f}{V_i}
\\
Third Law
& \text{At absolute zero of temperature, $S=0$ normally.}
\\
Entropy change on raising temperature at constant pressure
&
\Delta S = \int _{T_0}^{T_1} \frac{C_p}T\, dT
\approx
C_p\,\ln \frac{T_1}{T_0}
\\
Gibbs energy
%as a (negative) measure of universal entropy
&
\color{memorise}
G = H - T S
\\
% Reversible change, ie equilibrium
% &
% \delta S_{\text{universe}} = 0, \quad \delta G_{\text{system}} = 0
% \\
% Spontaneous change
% &
% \delta S_{\text{universe}} > 0, \quad \delta G_{\text{system}} < 0
% \\
Variation of $G$ with pressure, temperature and composition
&
\color{memorise}
dG=V\,dp-S\,dT+\mu_A\,dn_A+\mu_B\,dn_B+\mu_C\,dn_C+\dots
\\
% Variation of $G$ with temperature --
Gibbs-Helmholtz equation
&
\color{memorise}
 \left.\frac{ \partial(G/T) }{ \partial T}\right)_p = - \frac H{T^2}, \quad
 \left.\frac{ \partial(\Delta G/T) }{ \partial T}\right)_p = - \frac{\Delta H}{T^2}
\\
Simple integrated Gibbs-Helmholtz 
&
\frac{\Delta G_1}{T_1}
-\frac{\Delta G_0}{T_0}
= \Delta H \left(\frac{1}{T_1}-\frac{1}{T_0}\right)
\\
Variation of $G$ with $p$ for ideal gas
&
\Delta G = n\,R\,T\,\ln \frac{p_1}{p_0},\quad
G = G^\ltsym + n\,R\,T\,\ln\frac{p}{p^\ltsym}
\\
Phase rule
&
\color{memorise}
F = C - P + 2
\\
Clapeyron equation
&
%\color{memorise}
\frac{dp}{dT} = \frac{\Delta S_m}{\Delta V_m} = \frac{\Delta H_m }{ T\, \Delta V_m}
\\
Chemical potential of pure substance
&
\color{memorise}
\mu_{\text A} = G_{\text A,m}
\\
% Equilibrium in terms of chemical potential
% &
% \mu_{\text{reactant}} = \mu_{\text{product}}
% \\
Variation of $\mu$ with pressure for ideal gas
&
\color{memorise}
\mu = \mu^\ltsym + R\,T\,\ln \frac p{p^\ltsym}
\\
Ideal reaction quotient (gases)
&
\color{memorise}
Q = \frac
{
(p_{\text C}/p^\ltsym)^c
\,
(p_{\text D}/p^\ltsym)^d
}{
(p_{\text A}/p^\ltsym)^a
\,
(p_{\text B}/p^\ltsym)^b
}\text{ for }a\text A + b\text B \rightarrow c\text C+d\text D
\\
Variation of molar free energy change with conditions
&
\color{memorise}
\Delta G_m = \Delta G^\ltsym_m + R\,T\,\ln Q
\\
Equilibrium constant
% as the value of $Q$ at equilibrium ($\Delta G_m=0$)
&
\color{memorise}
\Delta G^\ltsym_m = - R\,T\,\ln K,\quad
K = e^{-\Delta G^\ltsym_m / R\,T}
% \\
% Activity
% &
% Generalisation of $p/p^\ltsym$ for components that are not ideal gases
\\
% Variation of equilibrium constant with temperature --
Van't Hoff equation
&
%\color{memorise}
\frac{d\ln K }{ d(1/T)} = -\frac{\Delta H^\ltsym_m }{ R} 
\\
Simple integrated Van't Hoff equation
&
\ln K = - \frac{\Delta H_m^\ltsym }{ R\ T} + \frac{\Delta S_m^\ltsym } R
\\
Raoult's law
&
\color{memorise}
p_A = x_A\, p_A^*, \quad
\mu_A=\mu_A^* + R\, T \ln x_A
\\
Ideal free energy and entropy of mixing
&
%\color{memorise}
\Delta_{\text{mix}}G=n R T (x_A \ln x_A+x_B\ln x_B)
\\ &
\Delta_{\text{mix}}S=-n R(x_A \ln x_A+x_B\ln x_B)
\\
Depression of
freezing point, elevation of boiling point
&
\delta T = \frac{x\, R\, T^2}{\Delta H_m}
\\
Osmotic pressure
&
\Pi = [B]\, R\, T
\end{concepts}
\section*{Electrochemistry}
\begin{concepts}
    Nernst equation &
    E_{\text{cell}}
    =
    E_{\text{cell}}^\ltsym - \frac{R\, T}{n\,F} \ln Q
\end{concepts}
\end{document}
%%% TeX-command-default: "PDFLaTeX"
%%% Local Variables: 
%%% mode: latex
%%% TeX-master: t
%%% End: 

